%!TEX program = xelatex
\documentclass[12pt]{beamer}
% \usetheme{Madrid}
\usetheme{Copenhagen}
% \usecolortheme{wolverine}

% Added because it is apparently useful for polyglossia
\usepackage{csquotes}

\usepackage[size=a4,scale=1.2]{beamerposter}
\usepackage{polyglossia}

\setbeamertemplate{headline}{}
\setbeamertemplate{footline}{}
\setbeamertemplate{frametitle}[default][center]

\setmainlanguage{english}
\usepackage{tikz}
\usepackage{fontspec}
\usepackage{booktabs}
\setlength{\tabcolsep}{.5em}
\usepackage{multirow}
\usepackage{bigdelim}
\usepackage{threeparttable}
\usefonttheme{serif}
\usepackage{noto}

\usepackage{multicol}
\setlength\multicolsep{0pt}
\setlength{\parskip}{0pt}

\makeatletter
\newcommand*{\textoverline}[1]{$\overline{\hbox{#1}}\m@th$}
\makeatother


\usepackage{lipsum}
\usepackage{forest}

\setlength{\leftmargini}{0.35cm}
\setlength{\leftmarginii}{0.25cm}

\newcommand{\nah}[1]{\textcolor{nahgrn}{#1}}
\newcommand{\trs}[1]{\textcolor{nahblu}{#1}}

\definecolor{nahred}{RGB}{224,32,13} 
\definecolor{nahyel}{RGB}{148,114,16} 
\definecolor{nahgrn}{RGB}{0,82,37} 
\definecolor{nahblu}{RGB}{0,93,197} 

\setbeamercolor{palette primary}{bg=nahred,fg=white}
\setbeamercolor{palette secondary}{bg=nahyel,fg=white}
\setbeamercolor{palette tertiary}{bg=nahgrn,fg=white}
\setbeamercolor{palette quaternary}{bg=nahblu,fg=white}
\setbeamercolor{structure}{fg=nahred} % itemize, enumerate, etc
\setbeamercolor{section in toc}{fg=nahred} % TOC sections


\usepackage[style=authortitle,isbn=false,backend=biber]{biblatex}
\addbibresource{nahuatlbib.bib}


\usepackage{verbatim} %In the preamble 

\title{Huasteca Nahuatl}
% \subtitle{cheat sheet}

\begin{document}

\begin{frame}%
  {Huasteca Nahuatl Cheatsheet}{Nouns, Adjectives, etc...}
  \begin{columns}[t]
    \begin{column}{.23\linewidth}
      \begin{block}{Nouns}
        \begin{enumerate}
			\item \begin{tabular}[t]{r|c|r|c}
			  \multicolumn{4}{l}{Noun \nah{Prefixes}} \\
			  pronoun 	& english 	&possession & english	\\
			  \cline{1-4}
			  \nah{ni-} & I			&\nah{no-}	& my		\\
			  \nah{ti-} & you		&\nah{mo-} 	& your 		\\
			  \nah{ø-}  & it		&\nah{i-} 	& its		\\
			  \nah{ti-} & we		&\nah{to-}	& our		\\
			  \nah{in-} & y'all		&\nah{inmo-}& y'all's	\\
			  \nah{ø-}  & they		&\nah{inin-}& their		\\
			\end{tabular}%
			\item \begin{tabular}[t]{c|c}
				\multicolumn{2}{l}{Absolutive \trs{Suffixes}} \\
				singular    & plural             	\\
				\cline{1-2}
				\trs{-tl}   & \trs{-meh}         \\
				\trs{-tli}  & \trs{-meh}         \\
				\trs{-li}   & \trs{-meh}         \\
				\trs{-n}    & \trs{-meh}         \\
				\trs{-ø}    & \trs{-meh}       	 \\
			\end{tabular}%
			\begin{tabular}[t]{c|c}
				\multicolumn{2}{l}{Reverential \trs{Suffixes}} \\
				singular    & plural             	\\
				\cline{1-2}
				\trs{-tzin}   & \trs{-tzitzin}         \\
				\trs{-tzin}   & \trs{-tzitzin}         \\
				\trs{-tzin}   & \trs{-tzitzin}         \\
				\trs{-tzin}   & \trs{-tzitzin}         \\
				\trs{-tzin}   & \trs{-tzitzin}         \\
			\end{tabular}%
			%\item \begin{tabular}[t]{lllllll}
			%	\multicolumn{7}{l}{Possession \nah{Prefixes} \& \trs{Suffixes}}    	   		       \\
			%	\multicolumn{3}{c}{singular} 			& \vline & \multicolumn{3}{c}{plural}	   \\
			%	\cline{1-7}
			%	\nah{no-} & noun & \trs{-ø} / \trs{-uh} & \vline & \nah{no-}   & noun & \trs{-huan}\\
			%	\nah{mo-} & noun & \trs{-ø} / \trs{-uh} & \vline & \nah{mo-} & noun & \trs{-huan}\\
			%	\nah{i-}  & noun & \trs{-ø} / \trs{-uh} & \vline & \nah{i-} & noun & \trs{-huan}\\
			%	\nah{to-} & noun & \trs{-ø} / \trs{-uh} & \vline & \nah{to-}   & noun & \trs{-huan}\\
			%	\nah{inmo-} & noun & \trs{-ø} / \trs{-uh} & \vline & \nah{inmo-} & noun & \trs{-huan}\\
			%	\nah{inin-}  & noun & \trs{-ø} / \trs{-uh} & \vline & \nah{inin-} & noun & \trs{-huan}\\
			%	\nah{te-} & noun & \trs{-ø} / \trs{-uh} & \vline & \nah{te-}   & noun & \trs{-huan}\\
			%\end{tabular}%
			\item \begin{tabular}[t]{lcccc}
				\multicolumn{5}{l}{Possession \nah{Prefixes} \& \trs{Suffixes} + \textit{Diminutive}}    	   		       \\
							&	  &		 & singular 						& plural	   \\
				\cline{1-5}
				\nah{no-} 	& \textit{pil-} & noun & \trs{-uh} / \trs{-ø} / \trs{hui} & \trs{-huan}\\
				\nah{mo-} 	& \textit{pil-} & noun & \trs{-uh} / \trs{-ø} / \trs{hui} & \trs{-huan}\\
				\nah{i-}  	& \textit{pil-} & noun & \trs{-uh} / \trs{-ø} / \trs{hui} & \trs{-huan}\\
				\nah{to-} 	& \textit{pil-} & noun & \trs{-uh} / \trs{-ø} / \trs{hui} & \trs{-huan}\\
				\nah{inmo-} & \textit{pil-} & noun & \trs{-uh} / \trs{-ø} / \trs{hui} & \trs{-huan}\\
				\nah{inin-} & \textit{pil-} & noun & \trs{-uh} / \trs{-ø} / \trs{hui} & \trs{-huan}\\
				\nah{te-} 	& \textit{pil-} & noun & \trs{-uh} / \trs{-ø} / \trs{hui} & \trs{-huan}\\
			\end{tabular}%
			\begin{example}
				\begin{tabular}{lllll}
					& B~1          & B~2          & B~3          &             \\
					Cl 1 & \nah{chōca-} & \nah{chōca-} & \nah{chōca-} & \trs{cry}   \\
					Cl 2 & \nah{yōli-}  & \nah{yōl-}   & \nah{yōli-}  & \trs{live}  \\
					Cl 3 & \nah{āltia-} & \nah{āltih-} & \nah{āltī-}  & \trs{bathe} \\
					Cl 4 & \nah{cua-}   & \nah{cuah-}  & \nah{cuā-}   & \trs{eat}
				\end{tabular}
			\end{example}
        \end{enumerate}
      \end{block}
	  \begin{block}{Relational Words} %TODO Found in Chapter 11
	  	empty for now
	  \end{block}
    \end{column}

    \begin{column}{.23\linewidth} %TODO add tla- new verb formation and incorporated nouns (chapter 10)
    	\begin{block}{Verb Tenses : Notes}
    		\textbf{verb} = \textbf{\textit{present}} tense verb \newline
    		\textbf{root} = present tense verb \textbf{\textit{without}} its final vowel \newline
    		\newline
    		\textbf{/k/} refers to the 'k' sound. Due to Spanish Orthography, we can't just use one letter. \newline
    		In the third person, the object prefix, however, is always "qui" \textbf{/ki/} \newline
    		\begin{enumerate}
    			\item Spanish Sound Explanation \newline
    			\begin{tabular}[t]{c|c|}
    				%\multicolumn{2}{l}{Spanish Sound Explanation} \\
    				spell & pronounce \\
    				\cline{1-2}
    				cu & \textbf{/kw/} 	\\
    				ce & \textbf{/se/} 	\\
    				ci & \textbf{/si/-} \\
    				ca & \textbf{/ka/} 	\\
    			\end{tabular}%
    			\begin{tabular}[t]{c|c}
    				%\multicolumn{2}{l}{Spanish Sound Explanation} \\
    				spell & pronounce \\
    				\cline{1-2}
    				qu & \textbf{/k/} \\
    				que & \textbf{/ke/} \\
    				qui & \textbf{/ki/}  \\
    				co & \textbf{/ko/} 	\\
    			\end{tabular}%
    			\newline
    			
    			\item Objects \newline
   				\begin{tabular}[t]{r|c|}
   					%\multicolumn{2}{l}{\textbf{Objects}} \\
   					object		   & english	\\
   					\cline{1-2}
   					\textbf{nech-} & me				\\
   					\textbf{mitz-} & you		 	\\
   					\textbf{/k/-}  & it				\\
   					\textbf{te-}  & someone		 	\\
   				\end{tabular}%
   				\begin{tabular}[t]{|r|c}
   					%\multicolumn{2}{l}{\textbf{Objects}} \\
   					object		   & english	\\
   					\cline{1-2}
   					\textbf{tech-} & us				\\
   					\textbf{mech-} & y'all			\\
   					\textbf{quin-} & them			\\
   					\textbf{tla-}  & thing		 	\\
   				\end{tabular}%
    		\end{enumerate}
    		
    		
    	\end{block}
    	\begin{block}{Verbs Tenses}
    		\begin{enumerate}
    			\item \begin{tabular}[t]{lllllll}
    				\multicolumn{7}{l}{\textbf{Present} \nah{Prefixes} \& \trs{Suffixes}}              \\
    				\multicolumn{3}{c}{singular}    & \vline & \multicolumn{3}{c}{plural}     \\
    				\cline{1-7}
    				\nah{ni-}   & verb & \trs{-ø}   & \vline & \nah{ti-}   & verb & \trs{-h}  \\
    				\nah{ti-}   & verb & \trs{-ø}   & \vline & \nah{in-}   & verb & \trs{-h}  \\
    				\nah{ø-}    & verb & \trs{-ø}   & \vline & \nah{ø-}    & verb & \trs{-h}  \\
    			\end{tabular}%
    			\item \begin{tabular}[t]{lllllll}
    				\multicolumn{7}{l}{\textbf{Future} \nah{Prefixes} \& \trs{Suffixes}}       	         \\
    				\multicolumn{3}{c}{singular}    & \vline & \multicolumn{3}{c}{plural}        \\
    				\cline{1-7}
    				\nah{ni-}   & root & \trs{-z}   & \vline & \nah{ti-}   & root & \trs{-zceh}  \\
    				\nah{ti-}   & root & \trs{-z}   & \vline & \nah{in-}   & root & \trs{-zceh}  \\
    				\nah{ø-}    & root & \trs{-z}   & \vline & \nah{ø-}    & root & \trs{-zceh}  \\
    			\end{tabular}%
    		\end{enumerate}
    	\end{block}
    	
    \end{column}

	%TODO The problem is in this column
    \begin{column}{.26\linewidth}
    	\begin{block}{Verbs Tenses : Past Tense}
    		
    		\begin{enumerate}
    			
    			\item \begin{tabular}[t]{l} %CLASS 1
    				Class \textbf{1} \nah{Prefixes} \& \trs{Suffixes}			\\
    				\textit{a). All monosyllabic verbs except cua and pa.} 		\\
    				\textit{b). Most, though not all verbs ending in -ca.} 		\\
    				\textit{c). All verbs ending in consonant-consonant-vowel.} \\
    				\textit{d). All verbs ending in -tla.} 						\\
    				\textit{e). All verbs ending in -o.}						\\
    			\end{tabular}
    			\begin{tabular}[t]{lllllll}
    				%\multicolumn{7}{l}{Class \textbf{1} \nah{Prefixes} \& \trs{Suffixes}}       	         \\
    				\multicolumn{3}{c}{singular}    & \vline & \multicolumn{3}{c}{plural}        \\
    				\cline{1-7}
    				\nah{ni-}   & verb & \trs{-c}   & \vline & \nah{ti-}   & verb & \trs{-queh}  \\
    				\nah{ti-}   & verb & \trs{-c}   & \vline & \nah{in-}   & verb & \trs{-queh}  \\
    				\nah{ø-}    & verb & \trs{-c}   & \vline & \nah{ø-}    & verb & \trs{-queh}  \\
    			\end{tabular}
    			
    			%CLASS 2
    			\item \text{Class \textbf{2} \nah{Prefixes} \& \trs{Suffixes}} \newline
    			\text{\textit{Most, but not all verbs ending in VCV.}}
    			\text{A consonant is always left at the end of the root when}
    			\text{the final vowel of the present tense form is eliminated.}
    			\newline
    			\begin{tabular}[t]{llll}
    				\multicolumn{3}{c}{Root Changes} & \vline \\
    				\cline{1-3}
    				hu	& > & uh & \vline	\\
    				cu	& > & uc & \vline	\\
    				y	& > & x	 & \vline	\\
    				qu	& > & c	 & \vline 	\\
    				c	& > & z	 & \vline 	\\
    				m	& > & n	 & \vline 	\\
    			\end{tabular}%
    			\begin{tabular}[t]{lllllll}
    				\multicolumn{3}{c}{singular}    & \vline & \multicolumn{3}{c}{plural}        \\
    				\cline{1-7}
    				\nah{ni-}   & root & \trs{-c}   & \vline & \nah{ti-}   & root & \trs{-queh}  \\
    				\nah{ti-}   & root & \trs{-c}   & \vline & \nah{in-}   & root & \trs{-queh}  \\
    				\nah{ø-}    & root & \trs{-c}   & \vline & \nah{ø-}    & root & \trs{-queh}  \\
    			\end{tabular}
    			%CLASS 3
    			\item Class \textbf{3} \nah{Prefixes} \& \trs{Suffixes} 			\\
    			\textit{All verbs ending in -ia or -oa.}							\\
    			\begin{tabular}[t]{lllllll}
    				%\multicolumn{7}{l}{Class \textbf{3} \nah{Prefixes} \& \trs{Suffixes}} 			\\
    				%\multicolumn{7}{1}{\textit{All verbs ending in -ia or -oa.}}					\\
    				\multicolumn{3}{c}{singular}    & \vline & \multicolumn{3}{c}{plural}        	\\
    				\cline{1-7}
    				\nah{ni-}   & root & \trs{-hqui}   & \vline & \nah{ti-}   & root & \trs{-hqueh} \\
    				\nah{ti-}   & root & \trs{-hqui}   & \vline & \nah{in-}   & root & \trs{-hqueh} \\
    				\nah{ø-}    & root & \trs{-hqui}   & \vline & \nah{ø-}    & root & \trs{-hqueh} \\
    			\end{tabular}%
    			%CLASS 4
    			\item Class \textbf{4} \nah{Prefixes} \& \trs{Suffixes} \\
    			\textit{Only pa, cua, mama, nahua, \& derivations.}		\\
    			\begin{tabular}[t]{lllllll} 
    				%\multicolumn{7}{l}{Class \textbf{4} \nah{Prefixes} \& \trs{Suffixes}}			\\
    				%\multicolumn{7}{1}{\textit{Only pa, cua, mama, nahua, \& derivations.}}			\\
    				\multicolumn{3}{c}{singular}    & \vline & \multicolumn{3}{c}{plural}        	\\
    				\cline{1-7}
    				\nah{ni-}   & verb & \trs{-hqui}   & \vline & \nah{ti-}   & verb & \trs{-hqueh} \\
    				\nah{ti-}   & verb & \trs{-hqui}   & \vline & \nah{in-}   & verb & \trs{-hqueh} \\
    				\nah{ø-}    & verb & \trs{-hqui}   & \vline & \nah{ø-}    & verb & \trs{-hqueh} \\
    			\end{tabular}
    			
    		\end{enumerate}
    		
    	\end{block}
    \end{column}
    \begin{column}{.20\linewidth}
      \begin{block}{Derivational suffixes}
      	\begin{threeparttable}
      		\begin{itemize}
      			\item \nah{-cauh}, pl.~\nah{-cahuān}  \trs{Vb-er}\tnote{1}
      			\item \nah{-huia} \trs{use N, provide N}
      			\item \nah{-(l)iztli} \trs{Vb-ing}
      			\item \nah{-lli} \trs{something Vb-ed}
      			\item \nah{-ni}, pl.~\nah{-nimeh}  \trs{Vb-er}\tnote{2}
      			\item \nah{-qui}, pl.~\nah{-queh}  \trs{Vb-er}\tnote{2}
      			\item \nah{-tia} \trs{provide with N, become N}
      			\item \nah{-yo} inalienable possession
      			\item \nah{-yoh} \trs{thing full of N}
      			\item \nah{-yōtl} \trs{N-ness, N-ship}
      		\end{itemize}
      		\begin{tablenotes}
      			\item[1] Only possessed nouns
      			\item[2] Only absolutive nouns
      		\end{tablenotes}
      	\end{threeparttable}
      \end{block}
      \begin{example}
      	\begin{itemize}
      		\item \nah{miqui} \trs{die}, \nah{miquiliztli} \trs{death}
      		\item \nah{nonac} \trs{my meat}, \nah{nonacayo} \trs{my flesh}
      		\item \nah{tlamati} \trs{he knows things}, \nah{tlamatini} \trs{(he is a) scholar}
      	\end{itemize}
      \end{example}
      \begin{block}{Postpositions}
        \begin{tabular}{ll}
          \nah{-c(o)}     & \trs{in, at}                         \\
          \nah{-ca}       & \trs{by means of, through, with}     \\
          \nah{-huān}     & \trs{(together) with, moreover, and} \\
          \nah{-īcampa}   & \trs{behind}                         \\
          \nah{-ihtic}    & \trs{inside, within}                 \\
          \nah{-īxpan}    & \trs{in front of, facing}            \\
          \nah{-nāhuac}   & \trs{next to, close to}              \\
          \nah{-pampa}    & \trs{because of, concerning}         \\
          \nah{-pan}      & \trs{on, in, at}                     \\
          \nah{-(t)icpac} & \trs{on top of}                      \\
          \nah{-tech}     & \trs{next to, adhering to}           \\
          \nah{-tlan}     & \trs{by, next to, below}             \\
          \nah{-tzīntlan} & \trs{below, underneath}              \\
        \end{tabular}%
        \begin{itemize}
          \item Postpositions are suffixed to posessive pronouns and nouns:
                \begin{itemize}
                  \item \nah{notech} \trs{next to me}
                  \item \nah{īhuan} \trs{with it}
                  \item \nah{tlalticpac} \trs{on the earth}
                \end{itemize}
          \item N-Postposition has a general indefinite meaning; \nah{ī/īm}-Postposition \nah{in} N has a specific definite meaning:
                \begin{itemize}
                  \item \nah{tepan} \trs{on stone(s)}
                  \item \nah{īpan in tetl} \trs{on the stone(s)}
                \end{itemize}
          \item Postpositions may take the reverential suffix \nah{-tzinco}
        \end{itemize}
      \end{block}
    \end{column}
  \end{columns}
\end{frame}


\begin{frame}%
	{Huasteca Nahuatl Cheatsheet}{Verbs}
	\begin{columns}[t]
		\begin{column}{.23\linewidth}
			\begin{block}{Nouns}
				\begin{enumerate}
					\item \begin{tabular}[t]{ll}
						% \multicolumn{3}{l}{\nah{Suffixes} \trs{be}} \\
						\multicolumn{2}{l}{Absolutive \trs{Suffixes}} \\
						singular    & plural             \\
						\trs{-tl}   & \trs{-meh}         \\
						\trs{-tli}  & \trs{-meh}         \\
						\trs{-li}   & \trs{-meh}         \\
						\trs{-n}    & \trs{-meh}         \\
						\trs{-ø}    & \trs{-meh}         \\
					\end{tabular}%
					\item \begin{tabular}[t]{lllllll}
						\multicolumn{7}{l}{Possession \nah{Prefixes} \& \trs{Suffixes}}    	   		       \\
						\multicolumn{3}{c}{singular} 			& \vline & \multicolumn{3}{c}{plural}	   \\
						\cline{1-7}
						\nah{no-} & noun & \trs{-ø} / \trs{-uh} & \vline & \nah{to-}   & noun & \trs{-huan}\\
						\nah{mo-} & noun & \trs{-ø} / \trs{-uh} & \vline & \nah{inmo-} & noun & \trs{-huan}\\
						\nah{i-}  & noun & \trs{-ø} / \trs{-uh} & \vline & \nah{inin-} & noun & \trs{-huan}\\
					\end{tabular}%
					\begin{example}
						\begin{tabular}{lllll}
							& B~1          & B~2          & B~3          &             \\
							Cl 1 & \nah{chōca-} & \nah{chōca-} & \nah{chōca-} & \trs{cry}   \\
							Cl 2 & \nah{yōli-}  & \nah{yōl-}   & \nah{yōli-}  & \trs{live}  \\
							Cl 3 & \nah{āltia-} & \nah{āltih-} & \nah{āltī-}  & \trs{bathe} \\
							Cl 4 & \nah{cua-}   & \nah{cuah-}  & \nah{cuā-}   & \trs{eat}
						\end{tabular}
					\end{example}
				\end{enumerate}
			\end{block}
		\end{column}
		
		\begin{column}{.26\linewidth}
			\begin{block}{Verbs Tenses}
				\textbf{verb} means the \textbf{\textit{present}} tense verb \newline
				\textbf{root} means the present tense verb \textbf{\textit{without}} its final vowel \newline
				\begin{enumerate}
					\item \begin{tabular}[t]{lllllll}
						\multicolumn{7}{l}{Present \nah{Prefixes} \& \trs{Suffixes}}              \\
						\multicolumn{3}{c}{singular}    & \vline & \multicolumn{3}{c}{plural}     \\
						\cline{1-7}
						\nah{ni-}   & verb & \trs{-ø}   & \vline & \nah{ti-}   & verb & \trs{-h}  \\
						\nah{ti-}   & verb & \trs{-ø}   & \vline & \nah{in-}   & verb & \trs{-h}  \\
						\nah{ø-}    & verb & \trs{-ø}   & \vline & \nah{ø-}    & verb & \trs{-h}  \\
					\end{tabular}%
					\item \begin{tabular}[t]{lllllll}
						\multicolumn{7}{l}{Future \nah{Prefixes} \& \trs{Suffixes}}       	         \\
						\multicolumn{3}{c}{singular}    & \vline & \multicolumn{3}{c}{plural}        \\
						\cline{1-7}
						\nah{ni-}   & root & \trs{-z}   & \vline & \nah{ti-}   & root & \trs{-zceh}  \\
						\nah{ti-}   & root & \trs{-z}   & \vline & \nah{in-}   & root & \trs{-zceh}  \\
						\nah{ø-}    & root & \trs{-z}   & \vline & \nah{ø-}    & root & \trs{-zceh}  \\
					\end{tabular}%
				
				\end{enumerate}
			\end{block}
			
		\end{column}
		\begin{column}{.20\linewidth}
			\begin{block}{Verbs Tenses : Past Tense}
				\textbf{verb} means the \textbf{\textit{present}} tense verb \newline
				\textbf{root} means the present tense verb \textbf{\textit{without}} its final vowel \newline
				\begin{enumerate}
					\item \begin{tabular}[t]{lllllll}
						\multicolumn{7}{l}{Class 1 \nah{Prefixes} \& \trs{Suffixes}}       	         \\
						\multicolumn{3}{c}{singular}    & \vline & \multicolumn{3}{c}{plural}        \\
						\cline{1-7}
						\nah{ni-}   & verb & \trs{-c}   & \vline & \nah{ti-}   & verb & \trs{-queh}  \\
						\nah{ti-}   & verb & \trs{-c}   & \vline & \nah{in-}   & verb & \trs{-queh}  \\
						\nah{ø-}    & verb & \trs{-c}   & \vline & \nah{ø-}    & verb & \trs{-queh}  \\
					\end{tabular}%
					\item \text{Class 2 \nah{Prefixes} \& \trs{Suffixes}}
					\text{A consonant is always left at the end of the root when}
					\text{the final vowel of the present tense form is eliminated.}
					\newline
					\begin{tabular}[t]{llll}
						\multicolumn{3}{c}{Root Changes} & \vline \\
						\cline{1-3}
						hu & > & uh & \vline	\\
						cu & > & uc & \vline	\\
						y & > & x 	& \vline	\\
						qu & > & c 	& \vline 	\\
						c & > & z 	& \vline 	\\
						m & > & n 	& \vline 	\\
					\end{tabular}%
					\begin{tabular}[t]{lllllll}
						\multicolumn{3}{c}{singular}    & \vline & \multicolumn{3}{c}{plural}        \\
						\cline{1-7}
						\nah{ni-}   & root & \trs{-c}   & \vline & \nah{ti-}   & root & \trs{-queh}  \\
						\nah{ti-}   & root & \trs{-c}   & \vline & \nah{in-}   & root & \trs{-queh}  \\
						\nah{ø-}    & root & \trs{-c}   & \vline & \nah{ø-}    & root & \trs{-queh}  \\
					\end{tabular}%
					\item \begin{tabular}[t]{lllllll}
						\multicolumn{7}{l}{Class 3 \nah{Prefixes} \& \trs{Suffixes}}       	         \\
						\multicolumn{3}{c}{singular}    & \vline & \multicolumn{3}{c}{plural}        \\
						\cline{1-7}
						\nah{ni-}   & root & \trs{-hqui}   & \vline & \nah{ti-}   & root & \trs{-hqueh}  \\
						\nah{ti-}   & root & \trs{-hqui}   & \vline & \nah{in-}   & root & \trs{-hqueh}  \\
						\nah{ø-}    & root & \trs{-hqui}   & \vline & \nah{ø-}    & root & \trs{-hqueh}  \\
					\end{tabular}%
					\item \begin{tabular}[t]{lllllll}
						\multicolumn{7}{l}{Class 4 \nah{Prefixes} \& \trs{Suffixes}}       	         \\
						\multicolumn{3}{c}{singular}    & \vline & \multicolumn{3}{c}{plural}        \\
						\cline{1-7}
						\nah{ni-}   & verb & \trs{-hqui}   & \vline & \nah{ti-}   & verb & \trs{-hqueh}  \\
						\nah{ti-}   & verb & \trs{-hqui}   & \vline & \nah{in-}   & verb & \trs{-hqueh}  \\
						\nah{ø-}    & verb & \trs{-hqui}   & \vline & \nah{ø-}    & verb & \trs{-hqueh}  \\
					\end{tabular}%
				\end{enumerate}
			\end{block}
		\end{column}
		\begin{column}{.23\linewidth}
			\begin{block}{Derivational suffixes}
				\begin{threeparttable}
					\begin{itemize}
						\item \nah{-cauh}, pl.~\nah{-cahuān}  \trs{Vb-er}\tnote{1}
						\item \nah{-huia} \trs{use N, provide N}
						\item \nah{-(l)iztli} \trs{Vb-ing}
						\item \nah{-lli} \trs{something Vb-ed}
						\item \nah{-ni}, pl.~\nah{-nimeh}  \trs{Vb-er}\tnote{2}
						\item \nah{-qui}, pl.~\nah{-queh}  \trs{Vb-er}\tnote{2}
						\item \nah{-tia} \trs{provide with N, become N}
						\item \nah{-yo} inalienable possession
						\item \nah{-yoh} \trs{thing full of N}
						\item \nah{-yōtl} \trs{N-ness, N-ship}
					\end{itemize}
					\begin{tablenotes}
						\item[1] Only possessed nouns
						\item[2] Only absolutive nouns
					\end{tablenotes}
				\end{threeparttable}
			\end{block}
			\begin{example}
				\begin{itemize}
					\item \nah{miqui} \trs{die}, \nah{miquiliztli} \trs{death}
					\item \nah{nonac} \trs{my meat}, \nah{nonacayo} \trs{my flesh}
					\item \nah{tlamati} \trs{he knows things}, \nah{tlamatini} \trs{(he is a) scholar}
				\end{itemize}
			\end{example}
			\begin{block}{Postpositions}
				\begin{tabular}{ll}
					\nah{-c(o)}     & \trs{in, at}                         \\
					\nah{-ca}       & \trs{by means of, through, with}     \\
					\nah{-huān}     & \trs{(together) with, moreover, and} \\
					\nah{-īcampa}   & \trs{behind}                         \\
					\nah{-ihtic}    & \trs{inside, within}                 \\
					\nah{-īxpan}    & \trs{in front of, facing}            \\
					\nah{-nāhuac}   & \trs{next to, close to}              \\
					\nah{-pampa}    & \trs{because of, concerning}         \\
					\nah{-pan}      & \trs{on, in, at}                     \\
					\nah{-(t)icpac} & \trs{on top of}                      \\
					\nah{-tech}     & \trs{next to, adhering to}           \\
					\nah{-tlan}     & \trs{by, next to, below}             \\
					\nah{-tzīntlan} & \trs{below, underneath}              \\
				\end{tabular}%
				\begin{itemize}
					\item Postpositions are suffixed to posessive pronouns and nouns:
					\begin{itemize}
						\item \nah{notech} \trs{next to me}
						\item \nah{īhuan} \trs{with it}
						\item \nah{tlalticpac} \trs{on the earth}
					\end{itemize}
					\item N-Postposition has a general indefinite meaning; \nah{ī/īm}-Postposition \nah{in} N has a specific definite meaning:
					\begin{itemize}
						\item \nah{tepan} \trs{on stone(s)}
						\item \nah{īpan in tetl} \trs{on the stone(s)}
					\end{itemize}
					\item Postpositions may take the reverential suffix \nah{-tzinco}
				\end{itemize}
			\end{block}
		\end{column}
	\end{columns}
\end{frame}
\end{document}
